\documentclass{article}

\usepackage[english]{babel}
\usepackage[letterpaper,top=2cm,bottom=2cm,left=3cm,right=3cm,marginparwidth=1.75cm]{geometry}
\usepackage[colorlinks=true, allcolors=blue]{hyperref}
\usepackage{float}
\usepackage{graphicx}
\usepackage{listings}


\title{Laboratorio 4 - Enlaces WAN}
\author{Martín Oscar Pavan}

\begin{document}
\maketitle
\section{Introducción}
Una conexión \textbf{punto a punto} permite interconectar 2 nodos mediante un canal dedicado, en cambio, una red \textbf{multipunto} el canal de datos se puede utilizar para comunicarse con diversos nodos. En las redes WAN, los protocolos deben mantener la sincronización adecuada para que los extremos se comuniquen correctamente.
\section{HDLC}
El protocolo HDLC(\textit{High-level Data Link Control}) es un protocolo de enlace de datos orientado a bits, usado en enlaces series sincrónicos.
\\Cisco desarrolló una versión propietaria \textbf{cHDLC} debido a que la versión estándar no soporta el transporte de múltiples protocolos en un mismo enlace. Esta versiín propietaria contiene un campo llamado \textit{protocol field} el cuál permite que un mismo enlace serial acomode distintos protocolos de la capa de red.
\\\\
Dicho protocolo es el default y se puede especificar su uso ejecutnado el siguiente comando:

\begin{lstlisting}
        Router(config-if)#encapsulation hdlc
\end{lstlisting}
\begin{figure}[H]
    \centering
    \includegraphics[width=0.5\linewidth]{routers.png}
    \caption{2 Routers interconectados}
    \label{fig:enter-label}
\end{figure}
En este escenario, nos encontramos una conexión serial entre dos routers en el cuál mediante el comando \textbf{show controller} verificaremos cuál es el extremo DCE(el encargado de proveer la señal de reloj) y cuál es el extremo DTE(encargado de recibir la señal de reloj). En nuestro caso, el extremo DCE es nuestro router1, por lo que vamos a configurar el reloj a 250000bps en modo \textit{interface} con el siguiente comando:
\begin{lstlisting}
    Router1(config-if)#clock rate 250000
\end{lstlisting}
También, podemos verificar que cada uno de los extremos tenga el protocolo correcto con el comando \textbf{show interfaces}.

\subsection{La trama cHDLC}
HDLC es un protocolo complejo definido mediante reglas de la norma ISO/IEC 7776:1995; pero cHDLC solo utiliza un subconjutno de ellas.
\\\\
Con la herramienta Wireshark, podemos observar el contenido de las tramas cHDLC del ping realizado entre los Routers de la Figura 1.
\\Podemos observar que el campo \textit{address} de los mensajes ICMP(ping) es 0x0f, indicando que se trata de un mensaje \textit{unicast}. Si se tratara de un mensaje \textit{multicast} sería 0x8f.
\\\\
Las tramas \textit{multicast} transportan mensajes SLARP(Serial Line Address Resolution Protocol) y las tramas \textit{unicast} los mensajes ICMP(ping).
\begin{itemize}
    \item Un mensaje SLARP sirve para intercambiar información sobre la configuración de las interfaces seriales en routers, permite configurar la conexión.
    \item La secuencia de los mensajes SLARP comienza con una solicitud \textit{request} que solicita información sobre el dispositivo adyacente. Luego, este último, genera un mensaje \textit{reply} que contiene la información solicitada. Por último, una vez establecida correctamente la conexión se envían periodicamente mensajes \textit{keepalive} que verifica que la conexión siga activa. 
    \item Aproximadamente cada 10 segundos se realiza el envío del \textit{keepalive}.
    \item El valor del campo \textit{protocol} de un mensaje SLARP es 0x8035
    \item El valor del campo \textit{protocol} de un mensaje \textit{unicast} es 0x0800
    \item La secuencia de los mensajes \textit{unicast} comienza con el envío de un \textit{Echo (ping) request} al otro dispositivo, este recibe el mensaje y responde con un \textit{Echo (ping) reply}. Este ida y vuelta permita verificar conectividad y latencia  entre los dispositivos.
\end{itemize}

\section{PPP}
El protocolo cHDLC solo puede ser utilizado en enlaces punto a punto de dispositivos Cisco.
\\Pero si intervienen otros dispositivos, conviene utilizar el protocolo PPP, cuál a diferencia de cHDLC nos otorga las siguientes características:
\begin{itemize}
    \item Autenticación mediante PPP, CHAP o EAP 
    \item Comprensión de datos
    \item Detección de errores
    \item Soporte multipunto
    \item Utilizado en redes bit o byte oriented
\end{itemize}
Utilizando el escenario de la figura 1, debemos utilziar el siguiente comando para que la interfaz utilice el protocolo PPP:
\begin{lstlisting}
    Router(config-if)#encapsulation ppp
\end{lstlisting} 
Verificamos que el protocolo cambio usando \textbf{show interfaces}. Ahora disponemos del protocolo LCP(Link Control Protocol) para establecimiento, configuración, negociación y testeo del enlace.
\\Para este caso se utilizan protocolos PPP LCP, los cuales tienen una fuunción similar a SLARP en cHDLC.
\\\\
Vamos a ensayar una forma de autenticación llamada PAP(Password Authentication Protocol). Primeramente, en la configuracion del terminal, habilitamos la password con el comando \textbf{enable password -password-} y luego definiremos los usuarios que negociarán en cada interface:
\begin{lstlisting}
    R0(config)#username R1 password cisco1
    R0(config)#interface serial 0/1/0
    R0(config-if)#ppp authentication pap
    R0(config-if)#ppp pap sent-username R0 password cisco0
\end{lstlisting}
De esta forma, desde R0 le estamos asignando el usuario y contraseña que R1 usará para autenticar a R0. Viceversa, puede realizarse la configración para establecer una conexión con el protocolo PPP y autenticación PAP.
\section{GitHub}
Puedes encontrar publicado en \href{https://github.com/martinpavan1/redes1}{mi GitHub} este pdf llamado PAVAN Redes I - TP5.

\end{document}
